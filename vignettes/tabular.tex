\documentclass[titlepage]{article}
\usepackage{listings}
\usepackage{xcolor,colortbl}
\usepackage{multirow}
\usepackage{courier}
\usepackage[colorlinks,linkcolor=blue,citecolor=blue,urlcolor=blue,breaklinks=true]{hyperref}
\lstset{basicstyle=\ttfamily\small , breaklines}
\usepackage[left=3cm,top=3cm,bottom=3cm, right=3cm,includehead,includefoot]{geometry}
\usepackage{fancyhdr,lastpage}
%\pagestyle{fancy}
\cfoot{Page \thepage\ of \pageref{LastPage}}
\fancyhfoffset{.25in}
\renewcommand{\headrulewidth}{0.25pt}
\renewcommand{\footrulewidth}{0pt}
\setlength{\headheight}{23pt}
\renewcommand{\labelitemiii}{$\circ$}
\usepackage{longtable}
\usepackage{amsmath}
\usepackage[T1]{fontenc}
\usepackage[scaled]{helvet}
\renewcommand*\familydefault{\sfdefault}
\usepackage{courier}
\usepackage{graphicx}
\usepackage{tocbibind}
\usepackage[parfill]{parskip}    % Activate to begin paragraphs with an empty line rather than an indent
\usepackage{upgreek}
\usepackage{textpos}
\usepackage{relsize}
\usepackage{upquote}
% Use \begin{landscape} and end{landscape} to rotate text %%%
\usepackage{pdflscape}
\usepackage{textcomp}
\usepackage{float}
\floatplacement{figure}{H}
\floatplacement{table}{H}
\usepackage[printonlyused,nohyperlinks]{acronym}
\def\bflabel#1{{\large#1\ \ \ \ }\hfill}
\usepackage{fixltx2e}
\setlength{\belowcaptionskip}{10pt}
\usepackage{Sweave}


%\VignetteIndexEntry{Formatting Tables with latexpdf}
\begin{document}
\Sconcordance{concordance:tabular.tex:C:/work/code/latexpdf/vignettes/tabular.Rnw:%
1 61 1 1 7 6 0 1 1 12 0 2 2 1 0 1 1 16 0 1 3 15 0 1 3 18 0 1 3 18 0 1 3 %
18 0 1 3 18 0 1 5 20 0 1 5 20 0 1 4 20 0 1 5 21 0 1 5 21 0 1 2 18 0 1 %
11 26 0 1 9 25 0 1 12 28 0 1 2}

\vspace*{2cm}
\begin{center}
\vspace{1.5cm}
{\Large Formatting Tables with latexpdf}\\
~\\
\today\\
~\\
Tim Bergsma\\
\end{center}
\newpage

\section*{Introduction}

The R package 'latexpdf' supports rich aesthetics for embedding tables in PDF documents, illustrated below.  Be sure to set chunk option 'results' to 'tex'.  Although illustrations us as.ltable(), many arguments are passed through to as.tabular().  See also as.pdf() for creating stand-alone table images.

\section*{Examples}
We make a sample data frame.
\begin{Schunk}
\begin{Sinput}
> x <- data.frame(
+   study=c(rep('PROT01',5),NA),
+   subject=rep(c(1001,1002),each=3),
+   time=c(0,1,2,0,1,2),
+   conc=c(0.12,34,5.6,.5,200,NA)
+ )
> x
\end{Sinput}
\begin{Soutput}
   study subject time   conc
1 PROT01    1001    0   0.12
2 PROT01    1001    1  34.00
3 PROT01    1001    2   5.60
4 PROT01    1002    0   0.50
5 PROT01    1002    1 200.00
6   <NA>    1002    2     NA
\end{Soutput}
\end{Schunk}
Now we try various invocations.
\begin{Schunk}
\begin{Sinput}
> library(latexpdf)
> writeLines(as.ltable(x))
\end{Sinput}
\begin{table}[H]
 \begin{center}
  \begin{tabular}{lrrr}
    \hline \hline
   study & subject & time & conc \\ \hline
   PROT01 & \verb#1001# & \verb#0# & \verb#0.12# \\
   PROT01 & \verb#1001# & \verb#1# & \verb#34.00# \\
   PROT01 & \verb#1001# & \verb#2# & \verb#5.60# \\
   PROT01 & \verb#1002# & \verb#0# & \verb#0.50# \\
   PROT01 & \verb#1002# & \verb#1# & \verb#200.00# \\
    & \verb#1002# & \verb#2# & \verb## \\ \hline
  \end{tabular}
 \end{center}
\end{table}\end{Schunk}
\begin{Schunk}
\begin{Sinput}
> writeLines(as.ltable(x,environments=NULL))
\end{Sinput}
\begin{table}[H]
 \begin{tabular}{lrrr}
   \hline \hline
  study & subject & time & conc \\ \hline
  PROT01 & \verb#1001# & \verb#0# & \verb#0.12# \\
  PROT01 & \verb#1001# & \verb#1# & \verb#34.00# \\
  PROT01 & \verb#1001# & \verb#2# & \verb#5.60# \\
  PROT01 & \verb#1002# & \verb#0# & \verb#0.50# \\
  PROT01 & \verb#1002# & \verb#1# & \verb#200.00# \\
   & \verb#1002# & \verb#2# & \verb## \\ \hline
 \end{tabular}
\end{table}\end{Schunk}
\begin{Schunk}
\begin{Sinput}
> writeLines(as.ltable(x,caption='Plasma Concentrations',label='pctab'))
\end{Sinput}
\begin{table}[H]
 \caption[Plasma Concentrations]{Plasma Concentrations \label{pctab}}
 \begin{center}
  \begin{tabular}{lrrr}
    \hline \hline
   study & subject & time & conc \\ \hline
   PROT01 & \verb#1001# & \verb#0# & \verb#0.12# \\
   PROT01 & \verb#1001# & \verb#1# & \verb#34.00# \\
   PROT01 & \verb#1001# & \verb#2# & \verb#5.60# \\
   PROT01 & \verb#1002# & \verb#0# & \verb#0.50# \\
   PROT01 & \verb#1002# & \verb#1# & \verb#200.00# \\
    & \verb#1002# & \verb#2# & \verb## \\ \hline
  \end{tabular}
 \end{center}
\end{table}\end{Schunk}
\begin{Schunk}
\begin{Sinput}
> writeLines(as.ltable(x,caption='Plasma Concentrations',cap.top=FALSE))
\end{Sinput}
\begin{table}[H]
 \begin{center}
  \begin{tabular}{lrrr}
    \hline \hline
   study & subject & time & conc \\ \hline
   PROT01 & \verb#1001# & \verb#0# & \verb#0.12# \\
   PROT01 & \verb#1001# & \verb#1# & \verb#34.00# \\
   PROT01 & \verb#1001# & \verb#2# & \verb#5.60# \\
   PROT01 & \verb#1002# & \verb#0# & \verb#0.50# \\
   PROT01 & \verb#1002# & \verb#1# & \verb#200.00# \\
    & \verb#1002# & \verb#2# & \verb## \\ \hline
  \end{tabular}
 \end{center}
 \caption[Plasma Concentrations]{Plasma Concentrations }
\end{table}\end{Schunk}
\begin{Schunk}
\begin{Sinput}
> writeLines(as.ltable(x,grid=TRUE,caption='grid is TRUE'))
\end{Sinput}
\begin{table}[H]
 \caption[grid is TRUE]{grid is TRUE }
 \begin{center}
  \begin{tabular}{l|r|r|r}
    \hline \hline
   study & subject & time & conc \\ \hline
   PROT01 & \verb#1001# & \verb#0# & \verb#0.12# \\ \hline
   PROT01 & \verb#1001# & \verb#1# & \verb#34.00# \\ \hline
   PROT01 & \verb#1001# & \verb#2# & \verb#5.60# \\ \hline
   PROT01 & \verb#1002# & \verb#0# & \verb#0.50# \\ \hline
   PROT01 & \verb#1002# & \verb#1# & \verb#200.00# \\ \hline
    & \verb#1002# & \verb#2# & \verb## \\ \hline
  \end{tabular}
 \end{center}
\end{table}\end{Schunk}
\begin{Schunk}
\begin{Sinput}
> writeLines(as.ltable(x,grid=TRUE,caption='Includes Walls',walls=1,rules=c(1,2,1)))
\end{Sinput}
\begin{table}[H]
 \caption[Includes Walls]{Includes Walls }
 \begin{center}
  \begin{tabular}{|l|r|r|r|}
    \hline
   study & subject & time & conc \\ \hline \hline
   PROT01 & \verb#1001# & \verb#0# & \verb#0.12# \\ \hline
   PROT01 & \verb#1001# & \verb#1# & \verb#34.00# \\ \hline
   PROT01 & \verb#1001# & \verb#2# & \verb#5.60# \\ \hline
   PROT01 & \verb#1002# & \verb#0# & \verb#0.50# \\ \hline
   PROT01 & \verb#1002# & \verb#1# & \verb#200.00# \\ \hline
    & \verb#1002# & \verb#2# & \verb## \\ \hline
  \end{tabular}
 \end{center}
\end{table}\end{Schunk}
\begin{Schunk}
\begin{Sinput}
> writeLines(as.ltable(x,grid=TRUE,caption='Custom Breaks',
+     colbreaks=c(0,2,0),rowgroups=x$subject
+ ))
\end{Sinput}
\begin{table}[H]
 \caption[Custom Breaks]{Custom Breaks }
 \begin{center}
  \begin{tabular}{lr||rr}
    \hline \hline
   study & subject & time & conc \\ \hline
   PROT01 & \verb#1001# & \verb#0# & \verb#0.12# \\
   PROT01 & \verb#1001# & \verb#1# & \verb#34.00# \\
   PROT01 & \verb#1001# & \verb#2# & \verb#5.60# \\ \hline
   PROT01 & \verb#1002# & \verb#0# & \verb#0.50# \\
   PROT01 & \verb#1002# & \verb#1# & \verb#200.00# \\
    & \verb#1002# & \verb#2# & \verb## \\ \hline
  \end{tabular}
 \end{center}
\end{table}\end{Schunk}
\begin{Schunk}
\begin{Sinput}
> writeLines(as.ltable(x,grid=TRUE,caption='Custom Justify',
+     numjust='left',charjust='right'
+ ))
\end{Sinput}
\begin{table}[H]
 \caption[Custom Justify]{Custom Justify }
 \begin{center}
  \begin{tabular}{r|l|l|l}
    \hline \hline
   study & subject & time & conc \\ \hline
   PROT01 & \verb#1001# & \verb#0# & \verb#0.12# \\ \hline
   PROT01 & \verb#1001# & \verb#1# & \verb#34.00# \\ \hline
   PROT01 & \verb#1001# & \verb#2# & \verb#5.60# \\ \hline
   PROT01 & \verb#1002# & \verb#0# & \verb#0.50# \\ \hline
   PROT01 & \verb#1002# & \verb#1# & \verb#200.00# \\ \hline
    & \verb#1002# & \verb#2# & \verb## \\ \hline
  \end{tabular}
 \end{center}
\end{table}\end{Schunk}
\begin{Schunk}
\begin{Sinput}
> writeLines(as.ltable(x,grid=TRUE,caption='Decimal Align',
+     justify=c('center','left','right','decimal')
+ ))
\end{Sinput}
\begin{table}[H]
 \caption[Decimal Align]{Decimal Align }
 \begin{center}
  \begin{tabular}{c|l|r|r}
    \hline \hline
   study & subject & time & conc \\ \hline
   PROT01 & \verb#1001# & \verb#0# & \verb#0.12# \\ \hline
   PROT01 & \verb#1001# & \verb#1# & \verb#34   # \\ \hline
   PROT01 & \verb#1001# & \verb#2# & \verb#5.6 # \\ \hline
   PROT01 & \verb#1002# & \verb#0# & \verb#0.5 # \\ \hline
   PROT01 & \verb#1002# & \verb#1# & \verb#200   # \\ \hline
    & \verb#1002# & \verb#2# & \verb## \\ \hline
  \end{tabular}
 \end{center}
\end{table}\end{Schunk}
\begin{Schunk}
\begin{Sinput}
> writeLines(as.ltable(x,grid=TRUE,caption='Not Verbatim',
+     justify=c('center','left','right','decimal'),
+     verbatim=FALSE
+ ))
\end{Sinput}
\begin{table}[H]
 \caption[Not Verbatim]{Not Verbatim }
 \begin{center}
  \begin{tabular}{c|l|r|r}
    \hline \hline
   study & subject & time & conc \\ \hline
   PROT01 & 1001 & 0 & 0.12 \\ \hline
   PROT01 & 1001 & 1 & 34    \\ \hline
   PROT01 & 1001 & 2 & 5.6  \\ \hline
   PROT01 & 1002 & 0 & 0.5  \\ \hline
   PROT01 & 1002 & 1 & 200    \\ \hline
    & 1002 & 2 &  \\ \hline
  \end{tabular}
 \end{center}
\end{table}\end{Schunk}
\begin{Schunk}
\begin{Sinput}
> writeLines(as.ltable(x,grid=TRUE,caption='Custom Column Width',
+     justify=c('center','left','right','decimal'),
+     colwidth=c(NA,NA,NA,'2cm')
+ ))
\end{Sinput}
\begin{table}[H]
 \caption[Custom Column Width]{Custom Column Width }
 \begin{center}
  \begin{tabular}{c|l|r|p{2cm}}
    \hline \hline
   study & subject & time & conc \\ \hline
   PROT01 & \verb#1001# & \verb#0# & \verb#0.12# \\ \hline
   PROT01 & \verb#1001# & \verb#1# & \verb#34   # \\ \hline
   PROT01 & \verb#1001# & \verb#2# & \verb#5.6 # \\ \hline
   PROT01 & \verb#1002# & \verb#0# & \verb#0.5 # \\ \hline
   PROT01 & \verb#1002# & \verb#1# & \verb#200   # \\ \hline
    & \verb#1002# & \verb#2# & \verb## \\ \hline
  \end{tabular}
 \end{center}
\end{table}\end{Schunk}
\begin{Schunk}
\begin{Sinput}
> writeLines(as.ltable(x,caption='Row Colors',rowcolors=c('white','lightgray')))
\end{Sinput}
\begin{table}[H]
 \caption[Row Colors]{Row Colors }
 \begin{center}
  \begin{tabular}{lrrr}
    \hline \hline
   study & subject & time & conc \\ \hline
   \rowcolor{white} PROT01 & \verb#1001# & \verb#0# & \verb#0.12# \\
   \rowcolor{lightgray} PROT01 & \verb#1001# & \verb#1# & \verb#34.00# \\
   \rowcolor{white} PROT01 & \verb#1001# & \verb#2# & \verb#5.60# \\
   \rowcolor{lightgray} PROT01 & \verb#1002# & \verb#0# & \verb#0.50# \\
   \rowcolor{white} PROT01 & \verb#1002# & \verb#1# & \verb#200.00# \\
   \rowcolor{lightgray}  & \verb#1002# & \verb#2# & \verb## \\ \hline
  \end{tabular}
 \end{center}
\end{table}\end{Schunk}
\begin{Schunk}
\begin{Sinput}
> writeLines(
+   as.ltable(
+     x,
+     caption='Row Groups',
+     rowgroups=as.character(x$subject),
+     rowgrouplabel='groups',
+     rowgrouprule = 2
+   )
+ )
\end{Sinput}
\begin{table}[H]
 \caption[Row Groups]{Row Groups }
 \begin{center}
  \begin{tabular}{l||lrrr}
    \hline \hline
   groups & study & subject & time & conc \\ \hline
   \multirow{3}{*}{1001} & PROT01 & \verb#1001# & \verb#0# & \verb#0.12# \\
                          & PROT01 & \verb#1001# & \verb#1# & \verb#34.00# \\
                          & PROT01 & \verb#1001# & \verb#2# & \verb#5.60# \\
   \multirow{3}{*}{1002} & PROT01 & \verb#1002# & \verb#0# & \verb#0.50# \\
                          & PROT01 & \verb#1002# & \verb#1# & \verb#200.00# \\
                          &  & \verb#1002# & \verb#2# & \verb## \\ \hline
  \end{tabular}
 \end{center}
\end{table}\end{Schunk}
\begin{Schunk}
\begin{Sinput}
> writeLines(
+   as.ltable(
+     x,
+     caption='Column Groups',
+     colgroups=c('demographic','demographic','clinical','clinical')
+   )
+ )
\end{Sinput}
\begin{table}[H]
 \caption[Column Groups]{Column Groups }
 \begin{center}
  \begin{tabular}{lrrr}
    \hline \hline
   \multicolumn{2}{c}{demographic} & \multicolumn{2}{c}{clinical} \\
   study & subject & time & conc \\ \hline
   PROT01 & \verb#1001# & \verb#0# & \verb#0.12# \\
   PROT01 & \verb#1001# & \verb#1# & \verb#34.00# \\
   PROT01 & \verb#1001# & \verb#2# & \verb#5.60# \\
   PROT01 & \verb#1002# & \verb#0# & \verb#0.50# \\
   PROT01 & \verb#1002# & \verb#1# & \verb#200.00# \\
    & \verb#1002# & \verb#2# & \verb## \\ \hline
  \end{tabular}
 \end{center}
\end{table}\end{Schunk}
\begin{Schunk}
\begin{Sinput}
> writeLines(
+   as.ltable(
+     x,
+     caption='Row and Column Groups',
+     rowgroups=as.character(x$subject),
+     colgroups=c('demographic','demographic','clinical','clinical'),
+     rowgrouprule = 1,
+     grid=TRUE
+   )
+ )
\end{Sinput}
\begin{table}[H]
 \caption[Row and Column Groups]{Row and Column Groups }
 \begin{center}
  \begin{tabular}{l|lr|rr}
    \hline \hline
   \multicolumn{1}{c|}{} & \multicolumn{2}{c|}{demographic} & \multicolumn{2}{c}{clinical} \\
     & study & subject & time & conc \\ \hline
   \multirow{3}{*}{1001} & PROT01 & \verb#1001# & \verb#0# & \verb#0.12# \\
                          & PROT01 & \verb#1001# & \verb#1# & \verb#34.00# \\
                          & PROT01 & \verb#1001# & \verb#2# & \verb#5.60# \\ \hline
   \multirow{3}{*}{1002} & PROT01 & \verb#1002# & \verb#0# & \verb#0.50# \\
                          & PROT01 & \verb#1002# & \verb#1# & \verb#200.00# \\
                          &  & \verb#1002# & \verb#2# & \verb## \\ \hline
  \end{tabular}
 \end{center}
\end{table}\end{Schunk}
\end{document}
